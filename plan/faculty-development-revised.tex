\documentclass[11pt,article,oneside]{memoir}
%BPR Tue, Dec 22, 2015 12:32: Crucially, this revised template uses article-5
% instead of article-4. I made article-5 to enable hyperlinks.

%% I prefer to see what I'm importing:
% \usepackage{org-preamble-xelatex}
%% Preamble/settings for documents exported from .org files to .tex files when the
%% tex engine is xelatex.
%%
%% Usage: \usepackage{org-preamble-xelatex} in your document
%% preamble. Font choice should be set in the document after the
%% package is declared. If \usepackage[minted]{org-preamble-xelatex}
%% is given, the minted package for code highlighting is turned
%% on. Minted requires that pygments be installed
%% (http://pygments.org/) and that \write18 support be enabled in your
%% tex compiler.

%% Kieran Healy
%%
\usepackage{endnotes}

\usepackage{varwidth}
\usepackage[T1]{fontenc}
% Fri, Sep 23, 2016 12:10:
% \catcode`\_=12

\setcounter{tocdepth}{4}

% \def\replace#1#2#3{%
%  \def\tmp##1#2{##1#3\tmp}%
%    \tmp#1\stopreplace#2\stopreplace}
% \def\stopreplace#1\stopreplace{}
% \usepackage{xstring}
\usepackage{xparse}

\ExplSyntaxOn
\NewDocumentCommand{\myreplace}{m}
 {
  \tl_set:Nn \l__maxd_argument_tl { #1 }
  \tl_replace_all:Nnn \l__maxd_argument_tl { _ } { . }
  \tl_use:N \l__maxd_argument_tl
 }
\tl_new:N \l__maxd_argument_tl
\ExplSyntaxOff

%Wed, Feb 10, 2016 10:32:
\usepackage[multidot]{grffile}

%Wed, Feb 10, 2016 10:32:
\newcommand{\bprspecialcell}[3][c]{%
  \begin{tabular}[#1]{@{}#2@{}}#3\end{tabular}}%

\usepackage{etoolbox}
\usepackage{listings}
%Added Tue, Jan 12, 2016 10:39
\usepackage{float}

%%Fri, Jan 8, 2016 10:54 Stata highlighting:
\usepackage{accsupp}% http://ctan.org/pkg/accsupp
\newcommand{\emptyaccsupp}[1]{\BeginAccSupp{ActualText={}}#1\EndAccSupp{}}

% langugae defination
\lstdefinelanguage{Stata}{
    % Left for users to add missing commands,
    % with possibility of choosing different style.
    morekeywords=,
    % Popular add-on commands
    morekeywords=[2]{cntrade, chinafin,
                     wbopendata, spmap,
    },
    % System commands
    morekeywords=[3]{regress, summarize,
                     display,
    },
    % Keywords
    morekeywords=[4]{forvalues, if, foreach, set, for, keep, drop, import, gen, egen, save, use, merge, rename, capture, rename, cd, capture, version, pause, on, program},
    % Built-in functions
    morekeywords=[5]{rnormal, runiform},
    morecomment=[l]{//},
    % morecomment=[l]{*},  // `*` maybe used as multiply operator. So use `//` as line comment.
    morecomment=[s]{/*}{*/},
    morecomment=[s]{*}{;},
    % morecomment=[s]{,}{//},
    % The following is used by macros, like `lags'.
    morecomment=[n][keywordstyle9]{`}{'},
    morestring=[b]",
    sensitive=true,
}

\lstdefinestyle{numbers}{
    numbers=left,
    stepnumber=1,
    numberstyle=\tiny\emptyaccsupp,
    xleftmargin=2em,
}

\lstdefinestyle{nonumbers}{
    numbers=none,
}

\lstdefinestyle{stata-plain}{
    % comment slanted and keywords bolded.
    language=Stata,
    basicstyle=\setmonofont{Consolas}\footnotesize\ttfamily,
}

\lstdefinestyle{stata-editor}{
    language=Stata,
    % size of the fonts for the code
    basicstyle=\setmonofont{Consolas}\footnotesize\ttfamily,
    % Color settings to match do-file editor style
    % Commands that are not included in the defination
    keywordstyle={\color{NavyBlue}},
    % Popular add-on commands
    keywordstyle=[2]{\color{NavyBlue}},
    % System commands
    keywordstyle=[3]{\color{NavyBlue}},
    % Keywords
    keywordstyle=[4]{\color{NavyBlue}},
    % Built-in functions like rnormal
    keywordstyle=[5]{\color{Blue}},
    % Used by macros, i.e. `lags'
    keywordstyle=[9]{\color{TealBlue}},
    stringstyle=\color{Maroon},
    commentstyle=\color{OliveGreen},
}


\lstset{
    language=Stata,
    style=stata-plain,
    % style=stata-editor,
    style=numbers,
    showstringspaces=false,
    breaklines,
    frame=single,
    % To add missing commands
    % morekeywords={xtreg, xtunitroot},
}

\newtoggle{rough}
\toggletrue{rough}

\newtoggle{draft}
\toggletrue{draft}
\togglefalse{draft}

\newtoggle{defcap}
\togglefalse{defcap}

\newtoggle{defnote}
\toggletrue{defnote}

\newtoggle{defwidth}
\toggletrue{defwidth}

\NeedsTeXFormat{LaTeX2e}
\ProvidesPackage{org-preamble-xelatex}[2011/02/21 v0.01 Bundling of Preamble items for Org to XeLaTeX export]

\RequirePackage{ifthen}
\newboolean{@minted} %

\setboolean{@minted}{false} % minted is off by default

\DeclareOption{minted}{
  \setboolean{@minted}{true}
}

\ProcessOptions

% Place below instead: Thu, Jan 7, 2016 3:17
% \RequirePackage{fontspec}
% \RequirePackage{xunicode}

\RequirePackage{url}
\RequirePackage{rotating}
\RequirePackage{memoir-article-styles}
\RequirePackage[american]{babel}
\RequirePackage[babel]{csquotes}
\RequirePackage[svgnames]{xcolor}
\RequirePackage{soul}
% \RequirePackage[xetex, colorlinks=true, urlcolor=DarkSlateBlue,
% citecolor=DarkSlateBlue, filecolor=DarkSlateBlue, plainpages=false,
% pdfpagelabels, bookmarksnumbered]{hyperref}
\RequirePackage[usenames,dvipsnames]{color}

% Incldued earlier:
% \RequirePackage{etoolbox}

%% Biblatex
% Modified Tue, Jan 12, 2016 10:25:
\usepackage[authordate,abbreviate=true,backend=biber,natbib]{biblatex-chicago}
% \RequirePackage[authordate, backend=biber, babel=hyphen, bibencoding=inputenc, strict, isbn=false, uniquename=false]{biblatex-chicago} % biblatex setup

% \RequirePackage[style=authoryear,
%             bibstyle=authoryear,
%             citestyle=authoryear,
%             %natbib=true,
%             hyperref=true,
%             backend=biber, babel=hyphen, bibencoding=inputenc]{biblatex}

%% Fix biblatex's odd preference for using In: by default.
\renewbibmacro{in:}{%
  \ifentrytype{article}{}{%
  \printtext{\bibstring{}\intitlepunct}}}

%% Basic bibliography
\addbibresource{/Users/brianreschke/Dropbox/bpr/Dissertation/bib/current.bib}
% \addbibresource{/Users/kjhealy/Documents/bibs/socbib.bib}
% \addbibresource{/Users/kjhealy/Documents/bibs/socbib-pandoc.bib}

%   %% these tweak the biblatex-chicago format to conform to AJS style.
   \DeclareFieldFormat[article]{title}{\mkbibquote{#1}}
   \DeclareFieldFormat[book]{title}{%
   \mkbibemph{#1}\isdot} % for books
\DeclareFieldFormat{booktitle}{\mkbibemph{#1}} % for edited volumes

%% bibnamedash: with Minion Pro the three-emdash lines in the
%% bibliogrpaphy end up separated from one another, which is very
%% annoying. Replace them with a line of appropriate size and weight.
%%\renewcommand{\bibnamedash}{\rule[3.5pt]{3em}{0.5pt}}

%% Pagestyle
\pagestyle{kjh}

%% If [minted] option is chosen, activate minted
\ifthenelse{\boolean{@minted}}{
  \RequirePackage{minted}
  \usemintedstyle{tango}
  \definecolor{bg}{rgb}{0.95,0.95,0.95}
  \setkeys{Gin}{width=1\@textwidth}
}{}

%\input{vc}
%\usepackage{inconsolata}



% Fri, Jan 8, 2016 10:27:
% 
% Added Tue, Jan 5, 2016 3:23:
\usepackage{caption}
\newcommand\fnote[1]{\captionsetup{font=scriptsize,justification=raggedright}\caption*{#1}}
\captionsetup{skip=0pt}
\newcommand\flnote[1]{\captionsetup{font=normalsize,justification=raggedright}\caption*{#1}}
% Thu, Sep 22, 2016 10:48 removed to allow for wider margins:
% \usepackage{geometry}
\usepackage[normalem]{ulem}
\usepackage[font=scriptsize]{subcaption}
\usepackage{amsmath}
\usepackage{mathtools}






% \title{\bigskip \bigskip Faculty Development Plan}
% \title{\bigskip \bigskip Faculty Development Plan}
\def\mytitle{\LARGE Faculty Development Plan}

%\author{true}

% \author{\Large Brian P. Reschke\vspace{0.05in} \newline\normalsize\emph{Department of Organizational Leadership and Strategy} \newline\footnotesize \url{brianreschke@byu.edu}\vspace*{0.2in}\newline }
\author{\Large Brian P. Reschke\vspace{0.05in} \newline\normalsize\emph{Department of Organizational Leadership and Strategy} \newline\footnotesize \protect\url{brianreschke@byu.edu}\vspace*{0.2in}\newline }

%\author{Brian P. Reschke (Department of Organizational Leadership and Strategy)}

\date{}

\usepackage{fontspec}
\usepackage{xunicode}
%BPR Tue, Dec 22, 2015 1:43: To specify font for section headings
\usepackage{titlesec}
\RequirePackage[xetex,
	colorlinks=true,
	urlcolor=DarkSlateBlue, % external links
        citecolor=DarkSlateBlue, % citations (setting affects links set by biber only, not pandoc-citeproc
        filecolor=DarkSlateBlue, % local files (Doesn't modify links: Tue, Dec 22, 2015 1:13)
        linkcolor=DarkSlateBlue, % local files
	plainpages=false,
  	pdfpagelabels,
  	bookmarksnumbered
    % hidelinks
  	]{hyperref}

\RequirePackage[all]{hypcap}

\newfontfamily\crimsonfont[Mapping=tex-text]{Crimson Text}

\newenvironment{cftabular}{\crimsonfont\tabular}{\endtabular}
\newenvironment{cflongtable}{\crimsonfont\longtable}{\endlongtable}

\begin{document}
\setkeys{Gin}{width=1\textwidth}
%% Crimson Text verion (before Tue, Aug 23, 2016 4:34)
% \setromanfont[Mapping=tex-text,Numbers=OldStyle]{Crimson Text}
% \setsansfont[Mapping=tex-text]{Crimson Text}
% \setmonofont[Mapping=tex-text,Scale=0.9]{Inconsolata}

% \setromanfont[Mapping=tex-text,Numbers=OldStyle]{Minion Pro}
% \setsansfont[Mapping=tex-text]{Minion Pro}
% \setmonofont[Mapping=tex-text,Scale=0.9]{Inconsolata}

\setmainfont{Minion Pro}[Mapping=tex-text,SmallCapsFeatures={LetterSpace=7},Numbers=OldStyle]
\setsansfont[Mapping=tex-text]{Minion Pro}
\setmonofont[Mapping=tex-text,Scale=0.9]{Inconsolata}
% \titleformat*{\section}{\Large\fontfamily{lmr}\selectfont\scshape\MakeLowercase}
\chapterstyle{article-7}
\pagestyle{kjh}
% Did this to avoid numbered chapters Thu, Jan 7, 2016 12:28:
\setcounter{secnumdepth}{-1}

\published{March 16, 2016.}

\title{\mytitle}
\maketitle



\begin{abstract}

\noindent The following plan was developed as part of the Faculty Development
Series at Brigham Young University. The intent of this plan is to aid my
development in scholarship, teaching, and citizenship over the coming
academic year, and to set a foundation for my continued contribution to
the mission and aims of BYU. For each professional area---scholarship,
teaching, and citizenship---I provide (a) a self-assessment of my
performance and abilities, and (b) goals and plans for the year.

\end{abstract}


\section{Scholarship}\label{scholarship}

\subsection{Self-Assessment}\label{self-assessment}

I study the interaction among social actors' various market signals and
actors' subsequent evaluation by external audiences. My work draws
principally from economic and organizational sociology, particularly the
literatures on status, commensuration, and categorization. In my work on
status, I study how an actor's sudden elevation in status alters the
attention a neighboring actor receives. In studies of market categories,
I investigate the causal effect of categorization on returns to coherent
and typical identity claims, as well as the potentially positive
consequences of novel category combinations. This work is unified in its
focus on the social nature of signals and the dynamic structure of
markets.

While exploring fundamental social science questions, my work also has
implications for managerial decision making. In many multi-sided
platforms (e.g., eBay, Alibaba, Facebook, Amazon, Airbnb, TaskRabbit,
eLance, Prosper, Kiva), managers are tasked with curating goods,
services, or content produced by users, as well as facilitating the
evaluation and selection of these products by others. My research speaks
directly to the \enquote{market-making} problems of optimally
categorizing offerings, facilitating comparison, and cultivating
positive cross-network effects.

In Winter 2016, I met with my department chair to discuss my progress in
scholarship. We agreed that while I have several projects in process,
and the work I have produced is of good quality, I can improve my
effectiveness in submitting projects for review and responding to
editorial decisions in a timely manner. I tend to continue tinkering
with manuscripts---exploring alternate modeling assumptions, measurement
approaches, and theoretical framing---with the (perhaps illusory) goal
of minimizing objections at the review stage. In discussion with
colleagues and from early career mentoring activities in the broader
academy, I see that a better approach is to prepare \enquote{minimum
viable papers} that capture the essence of what I am trying to do,
solicit preliminary review from collogues in the department and
elsewhere, incorporate this round of feedback, and submit these revised
\enquote{MVPs} for publication, then revising per the specific feedback
of reviewers.

\subsection{Goals and Plans}\label{goals-and-plans}

Below, I outline the goals for scholarship development through February
2017, and specify plans to achieve these goals. Where appropriate, I
describe current progress in achieving these goals.

\begin{itemize}
\tightlist
\item
  \textbf{Goal}: Submit a revised manuscript in response to a
  \enquote{revise and resubmit} decision

  \begin{itemize}
  \tightlist
  \item
    \textbf{Plans}: In mid-2015, I received a \enquote{revise and
    resubmit} decision on a first-authored manuscript from
    \emph{Administrative Science Quarterly}. My coauthors and I have
    through October 2016 to respond to reviewer comments. Responding to
    this R\&R has been top priority this summer. I recently completed
    all of the robustness check analyses requested, have drafted a new
    discussion section emphasizing our theoretical contributions, and am
    waiting on a coauthor's edits. My plan is to submit this manuscript
    in early September.
  \item
    \textbf{Principles}: \emph{Always under Review; \enquote{Furtherest
    along, First out} (FAFO)}: From a conversation with my mentor, I
    would like to have at least one paper under review at any given
    time. When I do not have a paper under review, I will prioritize the
    project closest to submission until I again have a paper under
    review. This discipline will help me appropriately evaluate my
    bandwidth when considering new projects.
  \end{itemize}
\item
  \textbf{Goal}: Submit two additional manuscripts for publication

  \begin{itemize}
  \tightlist
  \item
    \textbf{Plans}: In the course of winter semester's faculty
    development workshops, I joined a writing circle with five other
    colleagues in my department. I workshopped the above manuscript as
    well as a paper stemming from my dissertation. From participating in
    this circle, I have determined to consolidate two of my dissertation
    chapters into one paper investigating the causal effects of labeling
    on the returns of coherence and typicality. Though the writing
    circle disbanded at the start of summer, I will continue the
    practice of writing to a schedule, with a minimum of 30 minutes
    writing for this specific project daily through November 2016, by
    which time I intend to submit this manuscript for publication.
  \item
    \textbf{Principles}: Identify/create a writing exhibition deadline
    at least once every two weeks. I tend to not write otherwise. I will
    participate in writing circles that facilitate this frequency of
    sharing, and I will create others through sharing excerpts with my
    wife.
  \item
    \textbf{Plans}: At the recent Academy of Management meeting, I
    discussed past projects with a coauthor and we determined to revive
    and submit them for publication. One under consideration is a paper
    that was rejected from \emph{Management Science}, another (one we
    are more excited about) involves familiar data but would require new
    analysis and theoretical framing. Between these efforts, I intend to
    submit an additional manuscript for publication more in the manner
    of the \enquote{minimum viable paper} strategy described above. By
    mid-September, after my R\&R is completed, I will revisit these
    older projects and determine which would be most amenable to an MVP
    submission by February 2017.
  \end{itemize}
\item
  \textbf{Goal}: Submit two new, early-stage manuscripts for exhibition
  at major conferences

  \begin{itemize}
  \tightlist
  \item
    \textbf{Plans}: I am beginning two new projects, one with previous
    coauthors (Toby Stuart and Pierre Azoulay), and another with a new
    coauthor (Joel Baum). I will present my new project with Stuart and
    Azoulay at the West Coast Research Symposium in September. In
    October, I will present my project with Baum at the Network
    Evolution Conference at INSEAD. I will use both early exhibition
    opportunities to push manuscript development forward and to solicit
    feedback from key audiences. Then, in parallel with my
    \enquote{consolidation revision} described above, I will prepare
    conference manuscripts for the annual meetings of the Academy of
    Management and Strategic Management Society.
  \end{itemize}
\end{itemize}

\section{Teaching}\label{teaching}

\subsection{Self-Assessment}\label{self-assessment-1}

In Fall 2015, I taught two sections of Business Management 470:
Entrepreneurial Innovation, and in Winter 2016, I taught three sections
of Business Management 480R: New Venture Bootcamp. I had a wonderful
experience in my inaugural year of teaching. The students were a delight
to work with and I found the process of preparing for and facilitating
in-class learning experiences energizing and increasingly natural. One
of the overarching lessons from this first year was how doing the
learning activities I asked my students to do---even if my version of
the activity were necessarily abbreviated, given constraints---helped
bolster my integrity as an instructor and encouraged student engagement.
I was pleased to see that my qualitative self-assessment of how things
went was generally reflected in my student ratings: I was rated above
the department and university average.

Personally, I would like to improve in how I help students recognize
their own development, and I want to better sequence my class sessions.
Additionally, my student evaluations identify \enquote{Spiritually
Strengthening} and \enquote{Intellectually Enlarging} as the areas most
needing improvement. The latter teaching area relates to my learning
assessments and how I connect concepts over the course of the semester;
admittedly, \enquote{spiritually strengthening} seems more elusive to
me. Fortunately, sessions in the Faculty Development Seminar provided
several ideas on how I better invite the Spirit my classes. These ideas
are reflected in the goals and plans below.

\subsection{Goals and Plans}\label{goals-and-plans-1}

\begin{itemize}
\tightlist
\item
  \textbf{Goal}: Improve how my classes help students achieve the Aims
  of a BYU Education--particularly, improve how my course can be (a)
  spiritually strengthening and (b) intellectually enlarging. I will
  seek to assess my progress by qualitative feedback from students and
  the formal student ratings.

  \begin{itemize}
  \tightlist
  \item
    \textbf{Plans}: I will observe the teaching of two colleagues in
    courses related to my courses (Chad Carlos, MBA 671; Nile Hatch,
    Crocker Fellows) and ask them for advice on these specific areas. I
    will also invite two colleagues to observe my teaching in Fall 2016
    and develop a plan for improvement based on their feedback. Lastly,
    I will utilize the CTL mid-course review system and visit with them
    about the areas identified above.
  \end{itemize}
\end{itemize}

\section{Citizenship}\label{citizenship}

\subsection{Self-Assessment}\label{self-assessment-2}

Throughout the past year, I have established a pattern of citizenship
both inside and outside the university. I have participated in several
internal and external activities contributing to my university
citizenship. In Fall 2015, I served as a judge for the Big Idea
competition and the Student Entrepreneur of the Year competition, both
campus events. I also received an early external citizenship opportunity
in May 2016 as BYU convened the International Business Model Competition
(IBMC) at Microsoft headquarters in Redmond, Washington. There, I served
as a mentor for several competing teams from around the world and acted
as a judge in a semifinal round.

It seems there will be ample opportunities to serve the Center for
Entrepreneurship and Technology; I recognize that I could improve as a
citizen of my department and college, particularly in \emph{initiating}
informal discussions with colleagues and facilitating connections with
external scholars.

\subsection{Goals and Plans}\label{goals-and-plans-2}

\begin{itemize}
\tightlist
\item
  \textbf{Goal}: Provide service to the Rollins Center for
  Entrepreneurship and Technology (CET) as a judge or mentor in at least
  two campus-hosted competitions.

  \begin{itemize}
  \tightlist
  \item
    \textbf{Plans}: Attend the monthly correlation meetings of the CET
    to identify possible assignments.
  \end{itemize}
\item
  \textbf{Goal}: Take more initiative in arranging informal activities
  with my OLS and Marriott School colleagues.

  \begin{itemize}
  \tightlist
  \item
    \textbf{Plans}: Between now and February 2017, take the lead in
    organizing research conversations over lunch on at least four
    occasions, including at least once with colleagues outside my
    department.
  \end{itemize}
\item
  \textbf{Goal}: Invite two scholars from other universities to present
  in the OLS department research seminar, and assist in coordinating
  their schedules.

  \begin{itemize}
  \tightlist
  \item
    \textbf{Plans}: As this requires providing guest speakers advance
    notice, I will reach out to potential speakers this fall (likely
    during the conferences I am attending this semester) with the intent
    that they visit during the winter semester.
  \end{itemize}
\end{itemize}

\section{Resources}\label{resources}

The Office of the Dean has provided research support funds that will
facilitate the projects and conference travel described above. Already,
I have beneficially utilized the resources of the Center for Teaching
and Learning and the feedback of my colleagues. My group leader and
mentor has been an invaluable help in organizing my teaching and
engaging in citizenship assignments. Ultimately, the scarcest resource
will be my time. In my first year at BYU, I somewhat assembled my
professional development areas serially---with a teaching-focused fall
semester, followed by a research-focused winter semester, and
significant citizenship activities largely occurring at the end of the
academic year or in the summer. One meta-purpose of my faculty
development plan is to increase my capacity to thrive in my multiple
professional responsibilities simultaneously. I look forward to
describing my progress in subsequent reports.

\hypertarget{refs}{}


\end{document}
