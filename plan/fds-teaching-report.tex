\documentclass[11pt,article,oneside]{memoir}
%BPR Tue, Dec 22, 2015 12:32: Crucially, this revised template uses article-5
% instead of article-4. I made article-5 to enable hyperlinks.

%% I prefer to see what I'm importing:
% \usepackage{org-preamble-xelatex}
%% Preamble/settings for documents exported from .org files to .tex files when the
%% tex engine is xelatex.
%%
%% Usage: \usepackage{org-preamble-xelatex} in your document
%% preamble. Font choice should be set in the document after the
%% package is declared. If \usepackage[minted]{org-preamble-xelatex}
%% is given, the minted package for code highlighting is turned
%% on. Minted requires that pygments be installed
%% (http://pygments.org/) and that \write18 support be enabled in your
%% tex compiler.

%% Kieran Healy
%%
\usepackage{endnotes}

\usepackage{varwidth}
\usepackage[T1]{fontenc}
% Fri, Sep 23, 2016 12:10:
% \catcode`\_=12

\setcounter{tocdepth}{4}

% \def\replace#1#2#3{%
%  \def\tmp##1#2{##1#3\tmp}%
%    \tmp#1\stopreplace#2\stopreplace}
% \def\stopreplace#1\stopreplace{}
% \usepackage{xstring}
\usepackage{xparse}

\ExplSyntaxOn
\NewDocumentCommand{\myreplace}{m}
 {
  \tl_set:Nn \l__maxd_argument_tl { #1 }
  \tl_replace_all:Nnn \l__maxd_argument_tl { _ } { . }
  \tl_use:N \l__maxd_argument_tl
 }
\tl_new:N \l__maxd_argument_tl
\ExplSyntaxOff

%Wed, Feb 10, 2016 10:32:
\usepackage[multidot]{grffile}

%Wed, Feb 10, 2016 10:32:
\newcommand{\bprspecialcell}[3][c]{%
  \begin{tabular}[#1]{@{}#2@{}}#3\end{tabular}}%

\usepackage{etoolbox}
\usepackage{listings}
%Added Tue, Jan 12, 2016 10:39
\usepackage{float}

%%Fri, Jan 8, 2016 10:54 Stata highlighting:
\usepackage{accsupp}% http://ctan.org/pkg/accsupp
\newcommand{\emptyaccsupp}[1]{\BeginAccSupp{ActualText={}}#1\EndAccSupp{}}

% langugae defination
\lstdefinelanguage{Stata}{
    % Left for users to add missing commands,
    % with possibility of choosing different style.
    morekeywords=,
    % Popular add-on commands
    morekeywords=[2]{cntrade, chinafin,
                     wbopendata, spmap,
    },
    % System commands
    morekeywords=[3]{regress, summarize,
                     display,
    },
    % Keywords
    morekeywords=[4]{forvalues, if, foreach, set, for, keep, drop, import, gen, egen, save, use, merge, rename, capture, rename, cd, capture, version, pause, on, program},
    % Built-in functions
    morekeywords=[5]{rnormal, runiform},
    morecomment=[l]{//},
    % morecomment=[l]{*},  // `*` maybe used as multiply operator. So use `//` as line comment.
    morecomment=[s]{/*}{*/},
    morecomment=[s]{*}{;},
    % morecomment=[s]{,}{//},
    % The following is used by macros, like `lags'.
    morecomment=[n][keywordstyle9]{`}{'},
    morestring=[b]",
    sensitive=true,
}

\lstdefinestyle{numbers}{
    numbers=left,
    stepnumber=1,
    numberstyle=\tiny\emptyaccsupp,
    xleftmargin=2em,
}

\lstdefinestyle{nonumbers}{
    numbers=none,
}

\lstdefinestyle{stata-plain}{
    % comment slanted and keywords bolded.
    language=Stata,
    basicstyle=\setmonofont{Consolas}\footnotesize\ttfamily,
}

\lstdefinestyle{stata-editor}{
    language=Stata,
    % size of the fonts for the code
    basicstyle=\setmonofont{Consolas}\footnotesize\ttfamily,
    % Color settings to match do-file editor style
    % Commands that are not included in the defination
    keywordstyle={\color{NavyBlue}},
    % Popular add-on commands
    keywordstyle=[2]{\color{NavyBlue}},
    % System commands
    keywordstyle=[3]{\color{NavyBlue}},
    % Keywords
    keywordstyle=[4]{\color{NavyBlue}},
    % Built-in functions like rnormal
    keywordstyle=[5]{\color{Blue}},
    % Used by macros, i.e. `lags'
    keywordstyle=[9]{\color{TealBlue}},
    stringstyle=\color{Maroon},
    commentstyle=\color{OliveGreen},
}


\lstset{
    language=Stata,
    style=stata-plain,
    % style=stata-editor,
    style=numbers,
    showstringspaces=false,
    breaklines,
    frame=single,
    % To add missing commands
    % morekeywords={xtreg, xtunitroot},
}

\newtoggle{rough}
\toggletrue{rough}

\newtoggle{draft}
\toggletrue{draft}
\togglefalse{draft}

\newtoggle{defcap}
\togglefalse{defcap}

\newtoggle{defnote}
\toggletrue{defnote}

\newtoggle{defwidth}
\toggletrue{defwidth}

\NeedsTeXFormat{LaTeX2e}
\ProvidesPackage{org-preamble-xelatex}[2011/02/21 v0.01 Bundling of Preamble items for Org to XeLaTeX export]

\RequirePackage{ifthen}
\newboolean{@minted} %

\setboolean{@minted}{false} % minted is off by default

\DeclareOption{minted}{
  \setboolean{@minted}{true}
}

\ProcessOptions

% Place below instead: Thu, Jan 7, 2016 3:17
% \RequirePackage{fontspec}
% \RequirePackage{xunicode}

\RequirePackage{url}
\RequirePackage{rotating}
\RequirePackage{memoir-article-styles}
\RequirePackage[american]{babel}
\RequirePackage[babel]{csquotes}
\RequirePackage[svgnames]{xcolor}
\RequirePackage{soul}
% \RequirePackage[xetex, colorlinks=true, urlcolor=DarkSlateBlue,
% citecolor=DarkSlateBlue, filecolor=DarkSlateBlue, plainpages=false,
% pdfpagelabels, bookmarksnumbered]{hyperref}
\RequirePackage[usenames,dvipsnames]{color}

% Incldued earlier:
% \RequirePackage{etoolbox}

%% Biblatex
% Modified Tue, Jan 12, 2016 10:25:
\usepackage[authordate,abbreviate=true,backend=biber,natbib]{biblatex-chicago}
% \RequirePackage[authordate, backend=biber, babel=hyphen, bibencoding=inputenc, strict, isbn=false, uniquename=false]{biblatex-chicago} % biblatex setup

% \RequirePackage[style=authoryear,
%             bibstyle=authoryear,
%             citestyle=authoryear,
%             %natbib=true,
%             hyperref=true,
%             backend=biber, babel=hyphen, bibencoding=inputenc]{biblatex}

%% Fix biblatex's odd preference for using In: by default.
\renewbibmacro{in:}{%
  \ifentrytype{article}{}{%
  \printtext{\bibstring{}\intitlepunct}}}

%% Basic bibliography
\addbibresource{/Users/brianreschke/Dropbox/bpr/Dissertation/bib/current.bib}
% \addbibresource{/Users/kjhealy/Documents/bibs/socbib.bib}
% \addbibresource{/Users/kjhealy/Documents/bibs/socbib-pandoc.bib}

%   %% these tweak the biblatex-chicago format to conform to AJS style.
   \DeclareFieldFormat[article]{title}{\mkbibquote{#1}}
   \DeclareFieldFormat[book]{title}{%
   \mkbibemph{#1}\isdot} % for books
\DeclareFieldFormat{booktitle}{\mkbibemph{#1}} % for edited volumes

%% bibnamedash: with Minion Pro the three-emdash lines in the
%% bibliogrpaphy end up separated from one another, which is very
%% annoying. Replace them with a line of appropriate size and weight.
%%\renewcommand{\bibnamedash}{\rule[3.5pt]{3em}{0.5pt}}

%% Pagestyle
\pagestyle{kjh}

%% If [minted] option is chosen, activate minted
\ifthenelse{\boolean{@minted}}{
  \RequirePackage{minted}
  \usemintedstyle{tango}
  \definecolor{bg}{rgb}{0.95,0.95,0.95}
  \setkeys{Gin}{width=1\@textwidth}
}{}

%\input{vc}
%\usepackage{inconsolata}



% Fri, Jan 8, 2016 10:27:
% 
% Added Tue, Jan 5, 2016 3:23:
\usepackage{caption}
\newcommand\fnote[1]{\captionsetup{font=scriptsize,justification=raggedright}\caption*{#1}}
\captionsetup{skip=0pt}
\newcommand\flnote[1]{\captionsetup{font=normalsize,justification=raggedright}\caption*{#1}}
% Thu, Sep 22, 2016 10:48 removed to allow for wider margins:
% \usepackage{geometry}
\usepackage[normalem]{ulem}
\usepackage[font=scriptsize]{subcaption}
\usepackage{amsmath}
\usepackage{mathtools}






% \title{\bigskip \bigskip Course Development Project: Final Report}
% \title{\bigskip \bigskip Course Development Project: Final Report}
\def\mytitle{\LARGE Course Development Project: Final Report}

%\author{true}

% \author{\Large Brian P. Reschke\vspace{0.05in} \newline\normalsize\emph{Department of Organizational Leadership and Strategy} \newline\footnotesize \url{brianreschke@byu.edu}\vspace*{0.2in}\newline }
\author{\Large Brian P. Reschke\vspace{0.05in} \newline\normalsize\emph{Department of Organizational Leadership and Strategy} \newline\footnotesize \protect\url{brianreschke@byu.edu}\vspace*{0.2in}\newline }

%\author{Brian P. Reschke (Department of Organizational Leadership and Strategy)}

\date{}

\usepackage{fontspec}
\usepackage{xunicode}
%BPR Tue, Dec 22, 2015 1:43: To specify font for section headings
\usepackage{titlesec}
\RequirePackage[xetex,
	colorlinks=true,
	urlcolor=DarkSlateBlue, % external links
        citecolor=DarkSlateBlue, % citations (setting affects links set by biber only, not pandoc-citeproc
        filecolor=DarkSlateBlue, % local files (Doesn't modify links: Tue, Dec 22, 2015 1:13)
        linkcolor=DarkSlateBlue, % local files
	plainpages=false,
  	pdfpagelabels,
  	bookmarksnumbered
    % hidelinks
  	]{hyperref}

\RequirePackage[all]{hypcap}

\newfontfamily\crimsonfont[Mapping=tex-text]{Crimson Text}

\newenvironment{cftabular}{\crimsonfont\tabular}{\endtabular}
\newenvironment{cflongtable}{\crimsonfont\longtable}{\endlongtable}

\begin{document}
\setkeys{Gin}{width=1\textwidth}
%% Crimson Text verion (before Tue, Aug 23, 2016 4:34)
% \setromanfont[Mapping=tex-text,Numbers=OldStyle]{Crimson Text}
% \setsansfont[Mapping=tex-text]{Crimson Text}
% \setmonofont[Mapping=tex-text,Scale=0.9]{Inconsolata}

% \setromanfont[Mapping=tex-text,Numbers=OldStyle]{Minion Pro}
% \setsansfont[Mapping=tex-text]{Minion Pro}
% \setmonofont[Mapping=tex-text,Scale=0.9]{Inconsolata}

\setmainfont{Minion Pro}[Mapping=tex-text,SmallCapsFeatures={LetterSpace=7},Numbers=OldStyle]
\setsansfont[Mapping=tex-text]{Minion Pro}
\setmonofont[Mapping=tex-text,Scale=0.9]{Inconsolata}
% \titleformat*{\section}{\Large\fontfamily{lmr}\selectfont\scshape\MakeLowercase}
\chapterstyle{article-7}
\pagestyle{kjh}
% Did this to avoid numbered chapters Thu, Jan 7, 2016 12:28:
\setcounter{secnumdepth}{-1}

\published{March 20, 2016.}

\title{\mytitle}
\maketitle



\begin{abstract}

\noindent The following report describes my course development for BUS M 470:
Entrepreneurial Innovation.

\end{abstract}


\section{Course Background}\label{course-background}

I teach Business Management 470: Entrepreneurial Innovation. This is
among the first set of courses undergraduate entrepreneurship majors
take after being accepted into the major. Additionally, students from
other Marriott School of Management majors (strategy, organizational
behavior and human resources) take this course as an elective. The
purpose of the course is to help students \emph{engage in
entrepreneurship} in an intentional, systematic way, and to \emph{gain
confidence} in their ability to identify and innovate solutions for
problems. My course focuses on the generation and evaluation of ideas
for new ventures; from this course, the majors will go on to learn other
skills fundamental to actually launching new ventures.

While I have generally inherited most of the course content and
activities from a predecessor, this past Fall semester I conducted a
significant course revision. This was in part driven by the creation of
a new course that ran concurrently with BUS M 470 and that absorbed
about one third of the content I taught last Fall. This required and
allowed me to generate several new learning experiences, as well as to
allocate more time to some course Learning Objectives requiring more
emphasis.

\section{Learning Objectives and Relevant Course
Activities}\label{learning-objectives-and-relevant-course-activities}

The following section describes the Learning Objectives and associated
Course Activities for BUS M 470. Prior to teaching this course in Fall
2016, I revised these outcomes to better align with program Learning
Outcomes. These Learning Outcomes reflect the entrepreneurship major's
priority of developing critical thinking skills, as well as the
university's and Marriott School's emphasis on \enquote{Learn, Do,
Become}. For instance, the required and elective courses that follow
this course generally take as a given that students are able to supply a
creative new idea for a new venture: some of the electives require
visible progress on a business idea as a pre-requisite for admission.
Additionally, the outcomes reflect an orientation towards learning that
leads to action, and, ultimately, personal development---as students
learn systematic approaches to idea generation and design, they
immediately apply these lessons in course-long projects that are
regularly evaluated by me and their peers. In all, the outcomes
described below are well-suited for the course's timing relative to
student's entry to the major, and encourage a pedagogical strategy of
reinforcing content through application.

\subsection{Generate Creative New
Ideas}\label{generate-creative-new-ideas}

Cultivate your ability to think creatively and generate new ideas.

\begin{itemize}
\tightlist
\item
  Supports the Program Learning Objective of \enquote{Innovate} to solve
  real-world problems by increasing students' awareness of their world
  and considering ways it could be different.
\end{itemize}

\subsubsection{Relevant Course
Activities}\label{relevant-course-activities}

\begin{itemize}
\tightlist
\item
  Students keep an Idea Log throughout the entire semester in which they
  record and develop novel ideas for potential customer problems and
  associated solutions. They submit entries each week and also use the
  Log as a place to capture learning from some class exercises. Many
  have described this project as one of the best takeaways from the
  course, as it has spurred their development as critical thinkers. I
  also keep a Log and share insights periodically throughout the
  semester; this helps model certain patterns of creative thinking and
  provides a chance for students to connect with me on a personal level.
\item
  Students complete an online assessment of their Discovery and
  Execution behaviors. We use this as a springboard for discussing what
  creativity and innovation are, and how these attributes and activities
  are pliable and can be developed with practice. They also engage in a
  number of personal improvement projects in which they strive to do
  such things as ask more questions and seek new experiences. These
  activities broaden students' sources of new ideas, which in turn helps
  them participate in a higher rate of idea recombination.
\end{itemize}

\subsection{Understand Customers and Competitive
Contexts}\label{understand-customers-and-competitive-contexts}

Develop rich descriptions of customers' experiences with a specific
problem and the performance of existing solutions.

\begin{itemize}
\tightlist
\item
  Supports the Program Learning Objective of \enquote{Managing
  Uncertainty} by cultivating a deep understanding of sources of
  uncertainty in potential markets they may enter as entrepreneurs.
\end{itemize}

\subsubsection{Relevant Course
Activities}\label{relevant-course-activities-1}

\begin{itemize}
\tightlist
\item
  Prior to attending a certain session of class, students view an
  instructional video by Brene Brown distinguishing between
  \emph{sympathy} and \emph{empathy}. Students then design and execute a
  self-improvement project in which they intentionally engage in more
  empathetic behavior towards those around them. While applying empathy
  towards significant others (spouses, friends, roommates), they also
  acquire greater awareness of customers' challenges.
\item
  As a component of their second major term project, students design and
  implement two empathy experiences in which they try to simulate or
  directly experience the challenges of their customer of interest. They
  are then encouraged to identify a problem or pain of customers to
  solve based on learning from these experiences. Throughout the rest of
  the course, we emphasize the importance of understanding customers,
  particularly as we base solutions out of direct interaction with them
  and their problems.
\item
  Student groups participate in a \enquote{Domain Deep Dive} project, in
  which they are to exhaustively identify the existing solutions for the
  problem they are trying to solve.
\item
  We also lead the course with a case entitled \enquote{R\&R}, which
  permits a discussion of Porter's Five Forces, a tool for competitive
  analysis of industries. We return to this model in future discussions
  of market entry.
\end{itemize}

\subsection{Formulate and Prioritize
Hypotheses}\label{formulate-and-prioritize-hypotheses}

Identify and prioritize assumptions about customers, their problems, and
new solutions.

\begin{itemize}
\tightlist
\item
  Supports the Program Learning Objective of \enquote{Managing
  Uncertainty} by giving students a method of organizing their
  approaches to problem discovery and solution generation. By positing
  cause-and-effect relationships (hypothesizing) and then acquiring data
  to test these assumptions, students can see uncertainty is as
  manageable as our ability and resources to engage it scientifically.
\end{itemize}

\subsubsection{Relevant Course
Activities}\label{relevant-course-activities-2}

\begin{itemize}
\tightlist
\item
  As a class, we conduct several in-class activities that teach the
  principles of effective hypotheses and hypothesis testing. For
  example, in a discussion of a Ducati Corse Motorcycle case, we
  generate hypotheses for why Ducati failed to maintain their success
  rate in industry competitions; we also discuss how Ducati's approach
  to innovation did not facilitate hypothesis testing.
\item
  Additionally, students are asked to identify and justify the
  hypotheses that they examined in each of their project presentations.
\end{itemize}

\subsection{Create Elegant Solutions}\label{create-elegant-solutions}

Apply principles of design to create and validate elegant solutions to
customers' problems.

\begin{itemize}
\tightlist
\item
  Supports the Program Learning Objective of \enquote{Innovate
  Solutions} by facilitating students' direct participation in creating
  and applying solutions.
\end{itemize}

\subsubsection{Relevant Course
Activities}\label{relevant-course-activities-3}

\begin{itemize}
\tightlist
\item
  After a course lecture on principles of elegant solutions (Strategic
  Inventive Thinking), students conduct several in-class activities that
  allow them to apply the principles. For example, students were asked
  to bring a prototype of the solution their group was developing, and
  peers evaluated (among other things) the elegance of the proposed
  solution.
\item
  As part of their Solution pitch, student groups were required to
  evaluate the elegance of their proposed solution.
\end{itemize}

\subsection{Validate Entrepreneurial
Opportunities}\label{validate-entrepreneurial-opportunities}

Evaluate desirability, feasibilty, and viability of potential new
ventures through rigorous hypothesis testing.

\begin{itemize}
\tightlist
\item
  Supports the Program Learning Objective of \enquote{Create New
  Businesses} by supplying criteria whereby students can gauge whether a
  business opportunity should be pursued. In so doing, this objective
  supports the Program Learning Objective of \enquote{Manage
  Uncertainty}.
\end{itemize}

\subsubsection{Relevant Course
Activities}\label{relevant-course-activities-4}

\begin{itemize}
\tightlist
\item
  This outcome is best realized as students engage in the course-long
  group projects, in which they are assessing a potential business idea
  in terms of the demand for a solution, the feasibility of a solution,
  and profitability/sustainability of delivering a solution.
\item
  The final exam consists of a class presentation in which students
  synthesize the evidence of the semester and make an up or down
  decision on whether to pursue the entrepreneurial opportunity in
  subsequent months.
\end{itemize}

\section{Discussion of Course
Activities}\label{discussion-of-course-activities}

The course activities highlighted above demonstrate a mix of individual
projects and group assignments. This distribution of course activities
helps encourage personal application and initiative, while also
facilitating natural stretching through the presence of peers on team
projects. Thus, in addition to the feedback I provide on individual
assignments, students are receiving formal and informal feedback on
their efforts from peers on an even more frequent basis. The activities
also represent an assortment of learning types: daily
\enquote{catalysts} (pre-class activities or quizzes that spur in-class
discussion); case analysis and discussion; personal improvement projects
reported through essay responses; video supplements; discussion of
current events; and group projects. In encountering a variety of
instructional mediums, students experience course concepts and
principles in a manner that caters to diverse learning styles and also
have more opportunities for concept reinforcement. Additionally, as
described at the beginning of the previous section, I am continually
updating course activities in response to curriculum changes and student
feedback; I also strive to stay abreast of current events in
entrepreneurship and draw on the experiences of local startups for
\enquote{mini-case} discussions. I have also updated cases to reflect
more contemporary questions and challenges, while retaining
\enquote{classics} whose lessons are even more important in increasingly
turbulent environments. In these ways, I seek to provide students with
the most current thinking in the field, and I direct them to outlets and
habits that will help them continue to be so.

\section{Assessments of Student
Learning}\label{assessments-of-student-learning}

The assessments of student learning are intermingled with the activities
I conduct. As described in the previous section, they arrive at varying
frequency: for example, \enquote{catalysts} are administered as quizzes
before class sessions begin; as we debrief the quiz, we can naturally
assess what learning did and did not occur, and often I have altered
plans accordingly. The assessments are also timed to provide multiple
attempts at significant milestones. Particularly, there is not just one
final project presentation, but four spaced at three to four week
intervals. I also have a \enquote{practice} group project in which the
consequences for the grade are not as great as the four major project
presentations; several groups successfully re-tool after an enlightening
evaluation at this stage. Additionally, I incorporate students'
assessments of their peers in the weighting of group grades. This helps
bolster accountability to one another, and also allows me to assess what
groups are having interpersonal difficulties that are obstructing
learning.

The assessments are valid measures of learning. This is because they are
largely behavioral in nature. For example, the Learning Outcome
\enquote{Generate Creative Ideas} is assessed by means of the Idea Log
they submit weekly and in a culminating project at the end of the
semester. It is a stretch to stay consistent with this habit, but
students that complete it have clearly been applying themselves to their
environments, and many have described in personal conversations that
they are more generative than they have been before. (The degree to
which they are \enquote{creative} is hard to standardize, of course, but
by varying the prompts that I supply them, they are more likely to
ideate in new ways. See the Idea Log prompt in the Appendix.) Also, the
activity they engage in to Create Elegant Solutions or Validate
Entrepreneurial Activities generates artifacts in the forms of
prototypes, qualitative and quantitative responses, models of customer
behavior that did not exist prior to their efforts. These are activities
students and future employers can point to as evidence of ability. In
all, the behavioral nature of the course provides ample opportunity to
view assessment validity.

\section{Student Achievement of Learning
Outcomes}\label{student-achievement-of-learning-outcomes}

Below, I describe evidence that students have achieved the Learning
Outcomes.

\subsection{Generate Creative New
Ideas}\label{generate-creative-new-ideas-1}

Every student except one achieved a 90\% or higher on the final idea log
deliverable; that is, the vast majority of students were consistent in
their frequency of idea generation, and thus merited high credit on this
assessment. Had more students failed the assignments entirely---that is,
not generated ideas---there would be cause for concern that idea
generation was not happening as a result of this class.

\subsection{Understand Customers and Competitive
Contexts}\label{understand-customers-and-competitive-contexts-1}

The most direct assessment of students' ability to understand customers
is seen in their performance on the empathy development project. I was
quite surprised with the quality of students' efforts to develop empathy
with others---many students accumulated extra credit as a result of this
assignment. Particularly convincing is that all students except two
completed this project, thus expending considerable time and effort
directing their attention to others rather than themselves.

Additionally, a generally high score for each group on the presentation
components focusing on customer understanding suggests this Learning
Outcome is being reached.

\subsection{Formulate and Prioritize
Hypotheses}\label{formulate-and-prioritize-hypotheses-1}

While I am familiar with students' engagement in hypothesis development,
I do see that my present course assessments are sparser in the evidence
for this Learning Outcome. I will discuss plans for improvement for this
area in the section to follow.

\subsection{Create Elegant Solutions}\label{create-elegant-solutions-1}

One of the most convincing points of evidence for this learning outcome
is summarized in the students' performance on an in-class exercise
(evaluating a peer's prototype) they then reported on in a subsequent
turn-in assignment. Students did well on this assignment, with very few
achieving less than 80\% on the assignment. The qualitative evidence is
more striking: as I look back over student responses, I see that they
understand that elegant solutions can be achieved by applying the steps
of strategic inventive thinking (SIT). Students were detailed in
adhering to the steps of this paradigm, and they provide cogent
recommendations for how their peers' projects could improve. I saw
several groups modify their solutions based on the feedback generated by
this assignment.

\subsection{Validate Entrepreneurial
Opportunities}\label{validate-entrepreneurial-opportunities-1}

Evidence for this outcome is visible in the strong performance of almost
all groups in the final pitch assignment. This was the culminating
assignment of the semester, and students recognized that they were
supposed to take stock of all they had learned to that point and make a
clear decision on whether to pursue their ideas or not. I was impressed
with how students stepped up to the challenge---on the whole, they did
better than last year's cohort of entrepreneurship students. In
particular, students did well in making clear recommendations,
bolstering my confidence that they can similarly advise companies on new
product launch decisions or evaluate future entrepreneurial
opportunities. I feel that having an additional project milestone---the
Existing Solution Deep Dive---contributed to this stronger performance.

\section{Plans for Improvement}\label{plans-for-improvement}

As mentioned above, I see that the Learning Outcome \enquote{Formulate
and Prioritize Hypotheses}, while having several learning activities
allocated to it, the nature of the assessments does not provide clear
evidence one way or another (for myself or for students) that students
are mastering this topic. Accordingly, I will incorporate a new
assignment in the spirit of that for assessing SIT in which they have to
evaluate the quality of peers' hypotheses. This will not only provide
feedback to student groups that should improve their projects, but also
allow me to see where students may not understand effective hypothesis
generation.

Additionally, as a result of reading student comments and soliciting
feedback from colleagues who observed me, I would like to do better in
helping students see how each class session fits into the broader
picture of the course. Some students described enjoying the class
activities, but not really understanding the purpose behind it or how it
connected to course objectives. This is something I take seriously,
since Learning Objectives have little value if students are not able to
see the change in their own attitudes and behavior over the course of
the semester. I want to address this by an adjustment to my start of
class ritual: I want to have a slide up at the start of class as
students file in identifying (if appropriate--I may want to surprise
them) the agenda for the day and how it connects to the Learning
Objectives. I was pleased to see that students reported higher levels of
spiritual development and character building as a result of the class:
this was an area that I identified for improvement last year. I will
continue to interweave testimony and gospel principles organically in my
teaching.

\section{Conclusion}\label{conclusion}

In all, the process of preparing this course development report has
helped me see areas for course improvement. Particularly, it has helped
me see why the feedback on giving students more clear connection with
Learning Objectives is so important. I look forward to the next
opportunity to implement these plans for my development as a teacher.

\chapter{Appendix}\label{appendix}

\section{Idea Log Prompt}\label{idea-log-prompt}

What is your best new idea?~ Why is it a good idea?~ (See instructions
on the idea log below for features you may address to describe the
quality of an idea.)

Keep an idea log throughout the semester and record your ideas.~

\enquote{If you want to have good ideas you must have many ideas.~ Most
of them will be wrong and what you have to learn is which ones to throw
away.} (Linus Pauling).~ To have and keep more ideas, keep an idea log
throughout the semester.~ In addition to describing the idea, you may
want to describe what knowledge was recombined, the associative thinking
involved, target customers and their pain, possible solutions, and/or
market conditions.~ You will submit the idea log at the end of the
semester for me to evaluate.~ You may use any means to record your ideas
-- it does not have to be pen and paper.

Your idea log will be the place you will record and develop your many
ideas.~ Individually, you will deliver at least ten (10) novel ideas
using the tools of the course. (Ideally, you will discover far more than
ten novel ideas but I will evaluate ten.)~ You will submit your best
ideas in Gradebook and submit no more than one idea per week.~

You should be having, recording, and developing many more than ten novel
ideas.~ At the end of semester, you will submit your completed idea log
for evaluation.~ You may submit an electronic copy through LearningSuite
or a paper copy in class.~ You will already be getting credit for the
novel ideas during the semester so I will primarly be evaluating
the~other~ideas in your idea log.

My goal here is to help you build a habit that is one of the keys skills
of innovators. It is notoriously difficult to evaluate the quality or
novelty of an idea.~ I am not going to try.~ I will evaluate the habit
you are forming and the process you use to find, record, and refine your
ideas.

Criteria I use to evaluate your idea log:

\begin{itemize}
\tightlist
\item
  The number of ideas in addition to those submitted as weekly novel
  ideas.
\item
  The quality of ideas entered.~ Are they just simple statements of a
  cool product or thoughtful entries of pain and solutions that use the
  principles of the class?~ Are the top ideas being revisited over time
  as new evidence sheds new light on the ideas.
\item
  The pattern/timing of entries.~Is there evidence of a habit of
  entering ideas or were they entered in a rush at the end of the
  semester?
\item
  Were the ideas clearly dated?~ Are the submitted ideas clearly marked
  to distinguish them from non-submitted ideas?
\end{itemize}

Criteria I use to evaluate your submitted ideas:

\begin{itemize}
\tightlist
\item
  I will evaluate the novel ideas based primarily on your~ process for
  discovering them.~ I will probably not provide much feedback on all of
  your ideas.~ If you have a particularly good idea on which you want
  more extensive feedback, please let me know.~ In addition to the basic
  idea, you may include information such as (you do not need to include
  all of these facets, the list is here to remind of you things you may
  want to record about your idea)
\item
  Where did the idea come from?~ Which of our creativity tools generated
  your idea?~ What knowledge did you combine?~ What additional knowledge
  do you need?~ How will you get it?
\item
  What is the problem to be solved?~ Who has this problem?~ How deep is
  their pain?~ What are the existing solutions?~ Why is there residual
  pain?
\item
  What is your proposed solution for this pain?~ Is it likely to be
  adopted?~ Is it elegant?~ What does the solution landscape look like
  and how will you search it?
\item
  What is the market you are entering?~ What is the market size?~ Who
  are the competitors?~ What are your competitive advantages and how do
  they serve the customers' pain?
\item
  Why are you well-positioned to see and pursue this opportunity?~ What
  supplemental expertise or insight do you need to improve your odds?
\item
  How much value do you expect to be created?~ Who is likely to
  appropriate the value that is created?~ How much value are you likely
  to appropriate?
\end{itemize}

\hypertarget{refs}{}


\end{document}
