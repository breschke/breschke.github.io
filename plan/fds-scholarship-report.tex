\documentclass[11pt,article,oneside]{memoir}
%BPR Tue, Dec 22, 2015 12:32: Crucially, this revised template uses article-5
% instead of article-4. I made article-5 to enable hyperlinks.

%% I prefer to see what I'm importing:
% \usepackage{org-preamble-xelatex}
%% Preamble/settings for documents exported from .org files to .tex files when the
%% tex engine is xelatex.
%%
%% Usage: \usepackage{org-preamble-xelatex} in your document
%% preamble. Font choice should be set in the document after the
%% package is declared. If \usepackage[minted]{org-preamble-xelatex}
%% is given, the minted package for code highlighting is turned
%% on. Minted requires that pygments be installed
%% (http://pygments.org/) and that \write18 support be enabled in your
%% tex compiler.

%% Kieran Healy
%%
\usepackage{endnotes}

\usepackage{varwidth}
\usepackage[T1]{fontenc}
% Fri, Sep 23, 2016 12:10:
% \catcode`\_=12

\setcounter{tocdepth}{4}

% \def\replace#1#2#3{%
%  \def\tmp##1#2{##1#3\tmp}%
%    \tmp#1\stopreplace#2\stopreplace}
% \def\stopreplace#1\stopreplace{}
% \usepackage{xstring}
\usepackage{xparse}

\ExplSyntaxOn
\NewDocumentCommand{\myreplace}{m}
 {
  \tl_set:Nn \l__maxd_argument_tl { #1 }
  \tl_replace_all:Nnn \l__maxd_argument_tl { _ } { . }
  \tl_use:N \l__maxd_argument_tl
 }
\tl_new:N \l__maxd_argument_tl
\ExplSyntaxOff

%Wed, Feb 10, 2016 10:32:
\usepackage[multidot]{grffile}

%Wed, Feb 10, 2016 10:32:
\newcommand{\bprspecialcell}[3][c]{%
  \begin{tabular}[#1]{@{}#2@{}}#3\end{tabular}}%

\usepackage{etoolbox}
\usepackage{listings}
%Added Tue, Jan 12, 2016 10:39
\usepackage{float}

%%Fri, Jan 8, 2016 10:54 Stata highlighting:
\usepackage{accsupp}% http://ctan.org/pkg/accsupp
\newcommand{\emptyaccsupp}[1]{\BeginAccSupp{ActualText={}}#1\EndAccSupp{}}

% langugae defination
\lstdefinelanguage{Stata}{
    % Left for users to add missing commands,
    % with possibility of choosing different style.
    morekeywords=,
    % Popular add-on commands
    morekeywords=[2]{cntrade, chinafin,
                     wbopendata, spmap,
    },
    % System commands
    morekeywords=[3]{regress, summarize,
                     display,
    },
    % Keywords
    morekeywords=[4]{forvalues, if, foreach, set, for, keep, drop, import, gen, egen, save, use, merge, rename, capture, rename, cd, capture, version, pause, on, program},
    % Built-in functions
    morekeywords=[5]{rnormal, runiform},
    morecomment=[l]{//},
    % morecomment=[l]{*},  // `*` maybe used as multiply operator. So use `//` as line comment.
    morecomment=[s]{/*}{*/},
    morecomment=[s]{*}{;},
    % morecomment=[s]{,}{//},
    % The following is used by macros, like `lags'.
    morecomment=[n][keywordstyle9]{`}{'},
    morestring=[b]",
    sensitive=true,
}

\lstdefinestyle{numbers}{
    numbers=left,
    stepnumber=1,
    numberstyle=\tiny\emptyaccsupp,
    xleftmargin=2em,
}

\lstdefinestyle{nonumbers}{
    numbers=none,
}

\lstdefinestyle{stata-plain}{
    % comment slanted and keywords bolded.
    language=Stata,
    basicstyle=\setmonofont{Consolas}\footnotesize\ttfamily,
}

\lstdefinestyle{stata-editor}{
    language=Stata,
    % size of the fonts for the code
    basicstyle=\setmonofont{Consolas}\footnotesize\ttfamily,
    % Color settings to match do-file editor style
    % Commands that are not included in the defination
    keywordstyle={\color{NavyBlue}},
    % Popular add-on commands
    keywordstyle=[2]{\color{NavyBlue}},
    % System commands
    keywordstyle=[3]{\color{NavyBlue}},
    % Keywords
    keywordstyle=[4]{\color{NavyBlue}},
    % Built-in functions like rnormal
    keywordstyle=[5]{\color{Blue}},
    % Used by macros, i.e. `lags'
    keywordstyle=[9]{\color{TealBlue}},
    stringstyle=\color{Maroon},
    commentstyle=\color{OliveGreen},
}


\lstset{
    language=Stata,
    style=stata-plain,
    % style=stata-editor,
    style=numbers,
    showstringspaces=false,
    breaklines,
    frame=single,
    % To add missing commands
    % morekeywords={xtreg, xtunitroot},
}

\newtoggle{rough}
\toggletrue{rough}

\newtoggle{draft}
\toggletrue{draft}
\togglefalse{draft}

\newtoggle{defcap}
\togglefalse{defcap}

\newtoggle{defnote}
\toggletrue{defnote}

\newtoggle{defwidth}
\toggletrue{defwidth}

\NeedsTeXFormat{LaTeX2e}
\ProvidesPackage{org-preamble-xelatex}[2011/02/21 v0.01 Bundling of Preamble items for Org to XeLaTeX export]

\RequirePackage{ifthen}
\newboolean{@minted} %

\setboolean{@minted}{false} % minted is off by default

\DeclareOption{minted}{
  \setboolean{@minted}{true}
}

\ProcessOptions

% Place below instead: Thu, Jan 7, 2016 3:17
% \RequirePackage{fontspec}
% \RequirePackage{xunicode}

\RequirePackage{url}
\RequirePackage{rotating}
\RequirePackage{memoir-article-styles}
\RequirePackage[american]{babel}
\RequirePackage[babel]{csquotes}
\RequirePackage[svgnames]{xcolor}
\RequirePackage{soul}
% \RequirePackage[xetex, colorlinks=true, urlcolor=DarkSlateBlue,
% citecolor=DarkSlateBlue, filecolor=DarkSlateBlue, plainpages=false,
% pdfpagelabels, bookmarksnumbered]{hyperref}
\RequirePackage[usenames,dvipsnames]{color}

% Incldued earlier:
% \RequirePackage{etoolbox}

%% Biblatex
% Modified Tue, Jan 12, 2016 10:25:
\usepackage[authordate,abbreviate=true,backend=biber,natbib]{biblatex-chicago}
% \RequirePackage[authordate, backend=biber, babel=hyphen, bibencoding=inputenc, strict, isbn=false, uniquename=false]{biblatex-chicago} % biblatex setup

% \RequirePackage[style=authoryear,
%             bibstyle=authoryear,
%             citestyle=authoryear,
%             %natbib=true,
%             hyperref=true,
%             backend=biber, babel=hyphen, bibencoding=inputenc]{biblatex}

%% Fix biblatex's odd preference for using In: by default.
\renewbibmacro{in:}{%
  \ifentrytype{article}{}{%
  \printtext{\bibstring{}\intitlepunct}}}

%% Basic bibliography
\addbibresource{/Users/brianreschke/Dropbox/bpr/Dissertation/bib/current.bib}
% \addbibresource{/Users/kjhealy/Documents/bibs/socbib.bib}
% \addbibresource{/Users/kjhealy/Documents/bibs/socbib-pandoc.bib}

%   %% these tweak the biblatex-chicago format to conform to AJS style.
   \DeclareFieldFormat[article]{title}{\mkbibquote{#1}}
   \DeclareFieldFormat[book]{title}{%
   \mkbibemph{#1}\isdot} % for books
\DeclareFieldFormat{booktitle}{\mkbibemph{#1}} % for edited volumes

%% bibnamedash: with Minion Pro the three-emdash lines in the
%% bibliogrpaphy end up separated from one another, which is very
%% annoying. Replace them with a line of appropriate size and weight.
%%\renewcommand{\bibnamedash}{\rule[3.5pt]{3em}{0.5pt}}

%% Pagestyle
\pagestyle{kjh}

%% If [minted] option is chosen, activate minted
\ifthenelse{\boolean{@minted}}{
  \RequirePackage{minted}
  \usemintedstyle{tango}
  \definecolor{bg}{rgb}{0.95,0.95,0.95}
  \setkeys{Gin}{width=1\@textwidth}
}{}

%\input{vc}
%\usepackage{inconsolata}



% Fri, Jan 8, 2016 10:27:
% 
% Added Tue, Jan 5, 2016 3:23:
\usepackage{caption}
\newcommand\fnote[1]{\captionsetup{font=scriptsize,justification=raggedright}\caption*{#1}}
\captionsetup{skip=0pt}
\newcommand\flnote[1]{\captionsetup{font=normalsize,justification=raggedright}\caption*{#1}}
% Thu, Sep 22, 2016 10:48 removed to allow for wider margins:
% \usepackage{geometry}
\usepackage[normalem]{ulem}
\usepackage[font=scriptsize]{subcaption}
\usepackage{amsmath}
\usepackage{mathtools}






% \title{\bigskip \bigskip Scholarship Strategies Project: Final Report}
% \title{\bigskip \bigskip Scholarship Strategies Project: Final Report}
\def\mytitle{\LARGE Scholarship Strategies Project: Final Report}

%\author{true}

% \author{\Large Brian P. Reschke\vspace{0.05in} \newline\normalsize\emph{Department of Organizational Leadership and Strategy} \newline\footnotesize \url{brianreschke@byu.edu}\vspace*{0.2in}\newline }
\author{\Large Brian P. Reschke\vspace{0.05in} \newline\normalsize\emph{Department of Organizational Leadership and Strategy} \newline\footnotesize \protect\url{brianreschke@byu.edu}\vspace*{0.2in}\newline }

%\author{Brian P. Reschke (Department of Organizational Leadership and Strategy)}

\date{}

\usepackage{fontspec}
\usepackage{xunicode}
%BPR Tue, Dec 22, 2015 1:43: To specify font for section headings
\usepackage{titlesec}
\RequirePackage[xetex,
	colorlinks=true,
	urlcolor=DarkSlateBlue, % external links
        citecolor=DarkSlateBlue, % citations (setting affects links set by biber only, not pandoc-citeproc
        filecolor=DarkSlateBlue, % local files (Doesn't modify links: Tue, Dec 22, 2015 1:13)
        linkcolor=DarkSlateBlue, % local files
	plainpages=false,
  	pdfpagelabels,
  	bookmarksnumbered
    % hidelinks
  	]{hyperref}

\RequirePackage[all]{hypcap}

\newfontfamily\crimsonfont[Mapping=tex-text]{Crimson Text}

\newenvironment{cftabular}{\crimsonfont\tabular}{\endtabular}
\newenvironment{cflongtable}{\crimsonfont\longtable}{\endlongtable}

\begin{document}
\setkeys{Gin}{width=1\textwidth}
%% Crimson Text verion (before Tue, Aug 23, 2016 4:34)
% \setromanfont[Mapping=tex-text,Numbers=OldStyle]{Crimson Text}
% \setsansfont[Mapping=tex-text]{Crimson Text}
% \setmonofont[Mapping=tex-text,Scale=0.9]{Inconsolata}

% \setromanfont[Mapping=tex-text,Numbers=OldStyle]{Minion Pro}
% \setsansfont[Mapping=tex-text]{Minion Pro}
% \setmonofont[Mapping=tex-text,Scale=0.9]{Inconsolata}

\setmainfont{Minion Pro}[Mapping=tex-text,SmallCapsFeatures={LetterSpace=7},Numbers=OldStyle]
\setsansfont[Mapping=tex-text]{Minion Pro}
\setmonofont[Mapping=tex-text,Scale=0.9]{Inconsolata}
% \titleformat*{\section}{\Large\fontfamily{lmr}\selectfont\scshape\MakeLowercase}
\chapterstyle{article-7}
\pagestyle{kjh}
% Did this to avoid numbered chapters Thu, Jan 7, 2016 12:28:
\setcounter{secnumdepth}{-1}

\published{March 20, 2016.}

\title{\mytitle}
\maketitle



\begin{abstract}

\noindent The following report evaluates my progress towards goals for scholarly
development through February 2017.

\end{abstract}


\section{Goal Progress}\label{goal-progress}

In the course of the Faculty Development Series, I set three goals for
my scholarly development through the beginning of the Winter 2017
semester. As I review my progress towards each goal below, I identify
ways that applied the scholarly strategies identified in my Faculty
Development Plan. I also identify new insights gleaned from the
experience of reviewing my progress, and I articulate plans for future
productivity.

\subsection{\texorpdfstring{Goal: Submit a revised manuscript in
response to a \enquote{revise and resubmit}
decision}{Goal: Submit a revised manuscript in response to a revise and resubmit decision}}\label{goal-submit-a-revised-manuscript-in-response-to-a-revise-and-resubmit-decision}

I am pleased to report that I met the goal of successfully returning my
paper under first-round review at \emph{Administrative Science
Quarterly} to the reviewers. This proved more challenging than
anticipated, especially as a major coauthor encountered some unexpected
obstacles in the final stages of the revision. To work around this
challenge, I proceeded to revise a section formerly allocated to this
coauthor. This ultimately resulted in a new theoretical framework that
outgrew the original article---I have since incorporated a revision of
this framework in an invited article for \emph{Journal of Management
Inquiry}. This experience taught me the value of continuing to write,
even when there are available excuses to forestall writing. Even if not
incorporated in the intended piece, it may suggest paths for future
research. This insight came in the process of preparing this report: I
would like to apply this principle to early-stage projects that I have
delayed \enquote{writing up} because the research design is still in
flux.

\subsection{Goal: Submit two additional manuscripts for
publication}\label{goal-submit-two-additional-manuscripts-for-publication}

As mentioned, I submitted the first round of a manuscript to the
\emph{Journal of Management Inquiry} early this year. The article is
part of an invited series that may involve dialogue with other series
participants. Thus, I anticipate writing extensions to this manuscript
in the coming months. This experience will help establish my research
identity in the status literature. Unfortunately, I did not submit an
additional manuscript per my goals. However, the goal did help me spend
more time revising a past manuscript than I would have otherwise.
Particularly, per my (citizenship) strategy to engage more with my
colleagues, I workshopped this paper with members of my department
writing circle. Though the feedback was about the front end of the
paper---a section I highly anticipate revising as I get closer to
submission---soliciting feedback on this section helped me consider the
overall contribution of the paper. This made me realize that the
activities dominating my time on this revision (new analyses) were
probably outside the scope of this project, and that I would be better
served by completing the paper with the results I have in
hand---extending the analyses later as directed by reviewers.

I have learned from this experience that I still tend towards
\enquote{perfect drafts} at the expense of number of submissions.
Accordingly, I have revised my Faculty Development Plan to highlight the
principle of \enquote{Always under Review} and \enquote{Furtherest
along, First Out}, or FAFO. This was an insight stemming from a visit
with my faculty mentor. Particularly, this means that I would like to
have at least one paper under review at any given time. When I do not
have a paper under review, I will prioritize the project closest to
submission until I again have a paper under review (FAFO). This
discipline will help me appropriately evaluate whether I can take on new
projects.

One of the strategies I had identified in connection to this goal was to
write to schedule daily for at least 30 minutes. I observe that I do
much better at keeping with this program when I know there is a pending
opportunity to share my writing with others. Presently, this has been
around the time of submissions to conferences or journals, but I would
like to create many more \enquote{intermediate} writing exhibition
experiences for myself. Accordingly, I have revised my Faculty
Development Plan to recognize this principle of always having a writing
exhibition deadline in mind (whether a submission, conference, talk, or
draft exchange). I have resumed participation in the department writing
circle. This has helped my daily writing improve, though I see that the
size of the group limits the frequency in which I can solicit feedback.
My mentor and I have agreed to hold our own writing circle weekly moving
forward. This should provide more frequent accountability. I am excited
about the implications of these plans for my future productivity.

\subsection{Goal: Submit two new, early-stage manuscripts for exhibition
at major
conferences}\label{goal-submit-two-new-early-stage-manuscripts-for-exhibition-at-major-conferences}

Pursuant to my goal to \enquote{feed my pipeline}, I submitted an rough
manuscript reporting early results to the Annual Meeting of the Academy
of Management. The plan had been to submit a second project as well, but
the intended coauthors and I were occupied with the manuscript that
received the revise and resubmit described above. I see that timeliness
in turning around papers after editor decisions is not only important
for the focal paper, but also for fueling future research. I think the
above principle of striving to always have a work under review will
actually help me in identifying and developing new projects.

While I only partially met this goal, the manuscript I did submit was
part of a symposium I organized around a research topic related to my
future research stream. I identified participants through contacts I
made at a small conference at INSEAD, Fontainebleu, France. This last
week, I learned that the symposium was accepted by all three Academy
divisions from which I sought sponsorship; I am very hopeful about the
new collaborators that may come from participants and attendees of this
session. I also hope this session helps me take a more proactive role in
shaping a research conversation.

\hypertarget{refs}{}


\end{document}
